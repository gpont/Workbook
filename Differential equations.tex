\documentclass[a5paper,10pt]{article}

\usepackage{cmap}
\usepackage[a5paper,margin=1.5cm,noheadfoot]{geometry}
\usepackage[pdftex,unicode]{hyperref}
\usepackage[T2A]{fontenc}
\usepackage[utf8]{inputenc}
\usepackage[english,russian]{babel}
\usepackage{blindtext}
\usepackage{mathtools}
\usepackage{framed}


\def\letus{%
	\mathord{\setbox0=\hbox{$\exists$}%
		\hbox{\kern 0.125\wd0%
			\vbox to \ht0{%
				\hrule width 0.75\wd0%
				\vfill%
				\hrule width 0.75\wd0}%
			\vrule height \ht0%
			\kern 0.125\wd0}%
	}%
}

\author{Дмитрий Гутман, Евгений Гужихин}
\title{Алгоритмы решения контрольных по дифференциальным уравнениям, 2 курс, 1 и 2 семестр ФПМК}
\date{\today}

\begin{document}

	\maketitle

	\tableofcontents{}
	\newpage

	\section{I контрольная работа}
		\subsection{Однородное уравнение}
			\label{Homogeneous}
			\begin{enumerate}
				\item
					$$ y\prime=f(\frac{x}{y}) \eqno(1.1.1) $$ или $$ M(x,y)dx+N(x,y)dy=0 \eqno(1.1.2) $$,
					где $ M(x,y) $ и $ N(x,y) $ — однородные (т.е $ f(kx) = k^mf(x) $, где $ m $ — степень функции) функции одной и той же степени.
					\begin{enumerate}
						\item Сделать замену $ y = tx $;
						\item Решить, как уравнение с разделяющимися переменными.
					\end{enumerate}

				\item $$ y\prime=f\left(\cfrac{a_1x + b_1y + c_1}{ax + by + c}\right) \eqno(1.2) $$
					\begin{enumerate}
						\item если
							\begin{equation}
								\cfrac{a_1}{a_2}=\cfrac{b_1}{b_2}\neq\cfrac{c_1}{c_2}\nonumber,
							\end{equation}
							то $ z = a_1x + b_1y $ (или $ z = a_2x + b_2y $ или $ z = a_1x + b_1y + c_1 $ или $ z = a_2x + b_2y + c_2 $) и решить как уравнение с разделяющимися переменными.

						\item если
							\begin{equation}
								\cfrac{a_1}{a_2}=\cfrac{b_1}{b_2}=\cfrac{c_1}{c_2}\nonumber,
							\end{equation}
							то переменные уже можно разделить.
					\end{enumerate}
			\end{enumerate}

		\subsection{Уравнение Бернулли}
			\label{Bernulli}
			$$ y\prime + a(x)y = b(x)y^n, (n\neq1) \eqno(2) $$
			\begin{enumerate}
				\item Обе части разделить на $ y^n $;
				\item Сделать замену $ z = 1/y^{1-n} $, получится линейное дифференциальное уравнение I порядка \eqref{Linear}
			\end{enumerate}

			\subsubsection{Линейное дифференциальное уравнение I порядка}
				\label{Linear}
				$$ y\prime + a(x)y = f(x) \eqno(4) $$
				\begin{enumerate}
					\item Метод интегрирующего множителя
						\begin{enumerate}
							\item Найти интегрирующий множитель $ u(x) = e^{\int a(x)dx} $;
							\item $ y = \frac{\int u(x)f(x)dx + C}{u(x)} $.
						\end{enumerate}

					\item Метод вариации произвольно постоянной
						\begin{enumerate}
							\item Решить соответствующее однородное уравнение
							\item Сделать замену $ C $ на $ C(x) $, где $ C(x)-? $;
							\item Найти $ C(x) $;
							\item Подставить $ C(x) $ в исходное неоднородное.
						\end{enumerate}
				\end{enumerate}

		\subsection{Уравнение в полных дифференциалах}
			\label{Full_Diff}
			$$ F\prime_xdx + F\prime_ydy = 0 \eqno(3) $$
			\begin{enumerate}
				\item Доказать, что уравнение является уравнением в полных дифференциалах: $ F\prime\prime_{xy} = F\prime\prime_{yx} $;
				\item Восстановить $ F\prime_x \implies F = \int F\prime_xdx $, где $ C = \varphi(y) $ — некоторая пока неизвестная функция;
				\item Продифференцировать $ F $ по $ y $ $ (\varphi(y) \implies \varphi\prime_y(y)) $;
				\item Приравнять $ F\prime_y $ к результату выше и найти значение $ \varphi(y) $;
				\item Подставить $ \varphi(y) $ в $ F $ из пункта 2.
			\end{enumerate}

		\subsection{Замена $ y = z^m $}
			\begin{enumerate}
				\item Сделать замену $ y = z^m $;
				\item Найти $ m $ (сумма степеней всех слагаемых слева и справа должна совпадать).
			\end{enumerate}

		\subsection{Интегрирующий множитель}
			Некоторые методы определения интегрирующего множителя:
			\begin{enumerate}
				\item \underline{Метод последовательного выделения дифференциала.} Аналогичен методу выделения полного дифференциала для уравнений в полных дифференциалах. Только здесь полный дифференциал удается выделить, умножая уравнение на множители.
				\item \underline{Метод группировки членов уравнения.}
					\begin{enumerate}
						\item $ \letus p(x,y)dx + q(x,y)dy = 0 = (p_1dx + y_1dy) + p_2dx + q_2dy + \ldots, $
						\item $ \letus \exists M_1: M_1p_1dx + M_1q_1dy = dU_1. $
						\item Подбираем $ \phi(U_1): M_1\phi(U_1)(p_2dx + q_2dy) = dU_2, $
						\item $ \ldots $
					\end{enumerate}
				\item \underline{Определение интегрирующего множителя заданного вида.}


			\end{enumerate}

		\subsection{Уравнение Риккати}
			\label{Riccati}
			$$ y\prime + a(x)y + b(x)y^2 = c(x) \eqno(3) $$
			\begin{enumerate}
				\item Подбором найти частное решение $ y_1 $;
				\item Сделать замену $ y = y_1 + u $, получится уравнение Бернулли \eqref{Bernulli};
				\item Полученное уравнение можно привести к линейному \eqref{Linear} заменой $ z = \frac{1}{u} $.
			\end{enumerate}

		\newpage

	\section{II Контрольная работа}
		\subsection{Уравнение, неразрешенное относительно производной}
			$$ F(x,y,y\prime) = 0 $$
			Два способа решения:
			\begin{enumerate}
				\item Привести к $ y\prime = f_i(x,y), i=1,2,\ldots $. Получится одно или несколько уравнений такого вида, каждое нужно решить;
				
				\item Метод введения параметра:
					\begin{enumerate}
						\item Ввести параметр $ p = y\prime = \frac{dy}{dx} $, получим $ y = f(x,p) $;
						\item Взять полный дифференциал от обоих частей и заменить $ dy = pdx $, получим $ M(x,p)dx+N(x,p)dp=0 $.
					\end{enumerate}
				Если решение найдено в виде $ x=\phi(p) $, то получим $ x=\phi(p), y=f(\phi(p),p) $.
				\item $ y(x) = xy\prime+\psi(y\prime) $ - \underline{Уравнение Клеро}: Замена: $ y\prime:=p $.
				\item $ y(x) = x\phi(y\prime)+\psi(y\prime) $ - \underline{Уравнение Лагранжа}: $ y\prime:=p $.
			\end{enumerate}

		\subsection{Уравнение, неразрешенное относительно производной $n$-ого порядка}
			$$ F(x,y,y\prime,\ldots,y^{(n)}) = 0 $$
			Если:
			\begin{enumerate}
				\item Нет $ y,y\prime,\ldots,y^{(k-1)} \Rightarrow y^{(k)} := z(x) $
				
				\item Нет $ x \Rightarrow y\prime:=p(y), y\prime\prime:=p\prime*p, \ldots $
				
				\item $ \frac{d}{dx}\Phi(x, y, \ldots, y(n-1))=0 \Rightarrow (x, y, y\prime, \ldots, y^{(n-1)}) = C_1 \Rightarrow \ldots $
				
				\item Однородное относительно $ y,y\prime,\ldots,y^{(n)} $ (т.е. не меняется при замене $ y := ky, y\prime:=ky\prime, \ldots) \Rightarrow y\prime:=yz(x) $
				
				\item Обобщенно-однородное (т.е. не меняется при замене $ x:=kx, y:=k^m y, y\prime:=k^{m-1}y\prime, y\prime\prime:=k^{m-2}y\prime\prime, \ldots\Rightarrow x:=e^t, y:=z(t)e^{mt}, m - ? $
			\end{enumerate}

		\subsection{Линейное неоднородное уравнение с постоянными коэффициентами}
			\label{LinearNonhomogeneous}
			$$ \sum\limits_{i=0}^n a_i y^{(n-i)} = \phi(x): y_n(x) = y_h(x)+\hat y(x) $$, где $ y_n $ - решение неоднородного, $ y_h $ - решение соответствующего однородного, $ \hat y $ - частное решение
			\begin{enumerate}
				\item Решить соответствующее однородное $ y_h(x) - ? $: Замена $ y:=e^{kx} $

				\item Найти частные решения $ \hat y(x) - ? $
					\begin{enumerate}
						\item Метод Неопределенных Коэффициентов
							\begin{enumerate}
								\item Если $ \phi(x) =e^{px}(A_0x^s+\ldots+A_s) \Rightarrow $
									\begin{enumerate}
										\item Если $p$ - не корень характеристического уравнения $ \Rightarrow \hat y(x) = e^{px}(B_0x^s+\ldots+B_s) $, где $B_i$ - неопределенные коэффициенты
										\item Если $p$ - корень характеристического уравнения кратности $d \Rightarrow \hat y(x) = x^\alpha e^{px}(B_0x^s+\ldots+B_s)$, где $B_i$ - неопределенные коэффициенты
									\end{enumerate}
								\item Если $ \phi(x) = e^{px}(Q_s(x)cos(qx)+P_s(x)sin(qx)) $, где один из многочленов $Q_s$, $P_s$ имеет степень $s$, а другой $\leq s \Rightarrow$
									\begin{enumerate}
										\item Если $p\pm iq$ - не корни характеристического уравнения $ \Rightarrow \hat y(x) = e^{px}(\overline{Q_s}(x)cos(qx)+\overline{P_s}(x)sin(qx)) $
										\item Если $p\pm iq$ - корни характеристического уравнения $ \Rightarrow \hat y(x) = x^\alpha e^{px}(\overline{Q_s}(x)cos(qx)+\overline{P_s}(x)sin(qx)) $, где $Q_s$, $P_s$ - многочлены степени $s$ с неопределенными коэффициентами
									\end{enumerate}
							\end{enumerate}

						\item Метод вариации произвольных постоянных: решить СЛУ:
							\begin{equation*}
								\begin{cases}
									\sum\limits^n_{i=0}C\prime_i y_i = 0,\\
									\sum\limits^n_{i=0}C\prime_i y\prime_i = 0,\\
									\ldots,\\
									\sum\limits^n_{i=0}C\prime_i y^{n-2}_i = 0,\\
									\sum\limits^n_{i=0}C\prime_i y^{n-1}_i = \phi(x).
								\end{cases}
							\end{equation*}
					\end{enumerate}
			\end{enumerate}

		\subsection{Уравнение Эйлера}
			$$ x^ny^{(n)} + a_{n-1}x^{n-1}y^{(n-1)} + \ldots + a_1xy\prime + a_0y = 0 $$, где коэффициенты $ a_{n-1},\ldots,a_1,a_0 $ — постоянные действительные числа.
			Если функция $y(x)$ — решение уравнения Эйлера, то функция $Cy(x)$ тоже является решением уравнения.
			Уравнение Эйлера заменой $ x = e^t $ сводится к \underline{линейному однородному уравнению с постоянными коэффициентами} \ref{LinearNonhomogeneous}.

		\newpage

	\section{III контрольная работа}
		\subsection{Краевые задачи}
			$$ a_0(x)y\prime\prime + a_1(x)y\prime + a_2(x)y = f(x), \ x_0 \leq x \leq x_1 $$
			$$ \alpha y\prime(x_0) + \beta y(x_0) = 0, \ \gamma y\prime(x_1) + \delta y(x_1) = 0 $$
			\begin{enumerate}
				\item Решение краевой задачи
					\begin{enumerate}
						\item Найти общее решение уравнения;
						\item Подставить общее решение в краевые условия и определить значения произвольных постоянных, входящих в формулу общего решения.
					\end{enumerate}

				\begin{framed}
					\underline{Функция Грина $ G(x,s) $} - функция, определенная при $ x_0 \leq x \leq x_1, x_0 < s < x_1 $, и при каждом фиксированном $ s \in [x_0,x_1] $, обладающая свойствами:
					\begin{enumerate}
						\item при $ x \neq s $ она удовлетворяет уравнению
						$$ a_0(x)y\prime\prime + a_1(x)y\prime + a_2(x)y = f(x), \ x_0 \leq x \leq x_1; $$

						\item при $ x = x_0 $ и $ x = x_1 $ она удовлетворяет заданным краевым условиям;

						\item при $ x = s $ она непрерывна по $x$, а ее производная по $x$ имеет скачек, равный $ \frac{1}{a_0(s)} $, т.е.
						$$ G(s+1,s) = G(s-0,s), G_x\prime\bigg|_{x=s+0} = G_x\prime\bigg|_{x=s-0}+\frac{1}{a_0(s)}. $$
					\end{enumerate}
				\end{framed}

				\item Нахождение функции Грина:
					\begin{enumerate}
						\item Найти решения $ y_1(x), y_2(x) \neq 0 $, удовлетворяющих краевым условиям;
						\item Если $ y_1(x) $ не удовлетворяет сразу 2 краевым условиям, то
						\begin{equation*}
							\exists G(x,s) = 
							\begin{cases}
								ay_1(x) \ (x_0 \leq x \leq s),\\
								by_2(x) \ (s \leq x \leq x_1.
							\end{cases}
						\end{equation*}
						Функции $a,b$ зависят от $s$ и определяются из требования, чтобы функция $G$ удовлетворяла условиям: $ by_2(s) = ay_1(s), by_2\prime(s) = ay_1\prime(s) + \frac{1}{a_0(s)} $.
						\item Если $ \exists G $, то решение: $ y(x) = \int_{x_0}^{x_1} G(x,s)f(s)ds $.
					\end{enumerate}
			\end{enumerate}

		\subsection{Решение линейной неоднородной системы дифференциальных уравнений}
			$$ \frac{dx_i}{dt} = x\prime_i = \sum\limits_{j=1}^n a_{ij}x_j(t) + f_i(t), \; i = 1,2,\ldots,n, $$
			Или
			$$ \mathbf{X}'\left( t \right) = A\mathbf{X}\left( t \right) + \mathbf{f}\left( t \right), $$
			\begin{equation}
				\mathbf{X}\left( t \right) = \left(
					\begin{array}{*{20}{c}}
						{{x_1}\left( t \right)}\\
						{{x_2}\left( t \right)}\\
						 \vdots \\
						{{x_n}\left( t \right)}
					\end{array}
				\right), \;
				\mathbf{f}\left( t \right) = \left(
					\begin{array}{*{20}{c}}
						{{f_1}\left( t \right)}\\
						{{f_2}\left( t \right)}\\
						 \vdots \\
						{{f_n}\left( t \right)}
					\end{array}
				\right), \;
				A = \left(
					\begin{array}{*{20}{c}}
						{{a_{11}}}&{{a_{12}}}& \vdots &{{a_{1n}}}\\
						{{a_{21}}}&{{a_{22}}}& \vdots &{{a_{2n}}}\\
						 \cdots & \cdots & \cdots & \cdots \\
						{{a_{n1}}}&{{a_{n2}}}& \vdots &{{a_{nn}}}
					\end{array}
				\right),
			\end{equation}
			где $t$ - независимая переменная, $x_i(t)$ - неизвестные функции, $a_{ij}(\forall i,j=\overline{1,n})$ - постоянные коэффициенты, $f_i(t)$ - заданные функции. Функции $x_i(t)$, $f_i(t)$ и коэффициенты $a_{ij}$ могут принимать как действительные, так и комплексные значения. 
			\begin{framed}
				$ \mathbf{X}\left( t \right) = {\mathbf{X}_0}\left( t \right) + {\mathbf{X}_1}\left( t \right),$ где \par
				$ \mathbf{X}_0(t) $ - общее решение соответствующей однородной системы \par
				$ \mathbf{X}_1(t) $ - частное решения неоднородной системы.
			\end{framed}
			\begin{enumerate}
				\item Метод исключения \par
				(Удобно использовать для решения систем 2-го порядка.) Путем исключения неизвестных, система порядка $n$сводится к 1 уравнению порядка $n$ с 1 неизвестной функцией.

				\item Метод Эйлера(при помощи характеристического уравнения) \par
				\begin{enumerate}
					\item Найти собственные значения $ \lambda_i $ из характеристического уравнение системы: $ \det \left( {A - \lambda I} \right) = 0 $;

					\item Найти собственные вектора из уравнения $ \left( {A - {\lambda _i}I} \right){\mathbf{V}_i} = \mathbf{0} $;

					\item Решение: $ \mathbf{X}(t) = e^{\lambda t}\mathbf{V} $.
					\begin{framed}
						В случае $s_i = k_i = 1$, общее решение:
						\begin{equation}
							\mathbf{X}(t) = \left(
								\begin{array}{*{20}{c}}
									{{x_1}\left( t \right)}\\
									{{x_2}\left( t \right)}\\
									 \vdots \\
									{{x_n}\left( t \right)}
								\end{array}
							\right)
							= \sum\limits_{i=1}^n C_ie^{\lambda_i}t\mathbf{V}_i,
						\end{equation}
						где $ C_i $ - произвольные константы, $s_i$ - геометрическая кратность $\lambda_i$ (число собственных векторов, ассоциированных с $\lambda_i$), $k_i$ - алгебраическая кратность $\lambda_i$, $ 0 < s_i \leq k_i $.
					\end{framed}
				\end{enumerate}
			\end{enumerate}

		\subsection{Решение однородной системы матричным методом}


\end{document}