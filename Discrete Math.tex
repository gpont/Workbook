\documentclass[a5paper,10pt]{article}

\usepackage{cmap}
\usepackage[a5paper,margin=1.5cm,noheadfoot]{geometry}
\usepackage[T2A]{fontenc}
\usepackage[utf8]{inputenc}
\usepackage[english,russian]{babel}
\usepackage{blindtext}
\usepackage{mathtools}

\author{Евгений Гужихин}
\title{Дополнительные главы дискретной математики(конспект), 2 курс, 2 семестр ФПМК}
\date{\today}

\begin{document}

	\maketitle
	\tableofcontents{}
	\newpage

	\section{1 лекция}

	\section{2 лекция}
		\subsection{}
			\paragraph{}
				Автомат называется приведенным, если любые его состояния отличимы, то есть неэквивалентны.
				Если состояние $q_i$ не эквивалентно $q_j$, то $\exists$ некая последовательность $\beta: \phi(\beta_{q_i})=\phi(\beta_{q_j})$.
				2 автомата $a_1$ и $a_2$ называются эквивалентными, если для $\forall$ состояния 1 автомата $\exists$ эквивалентное состояние 2 автомата.
				Задача минимизация числа состояний в (полностью определенном) конечном автомате сводится к построению для него приведенного автомата, то есть автомата с минимальным числом состояний.
				Задачу сведем к поиску разбиений состояний заданного автомата А на классы эквивалентностей.

			\paragraph{}
				Алгоритм поиска разбиения состояний на классы эквивалентностей:
				(Будем считать, что автомат $A$ задан таблицами переходов-выходов
				$ A<X,Q,Y,\Phi,\Psi> \rightarrow \hat A $)
				Анализируя таблицу выходов, разобьем множество q на подмножество след. образом:
					В подмножество входит состояние, для которых столбцы таблицы выходов одинаковы.
					$P_1$
					...
					$P_k$
					Выбераем очередной блок из $P_k$ и состояние $q_i, q_j$ относим к одному и тому же блоку в разбиении $P_{k+1}$, если для $\forall$ $x \in X, \Psi(x,q_i),\Psi(x,q_j) \in P_k$
					...
					$P_s$ - невозможно дальше изменить. $P_s$ - искомое разбиение на классы эквивалентностей

			\paragraph{}
				\underline{Пример:} \\
				\begin{table}[h]
					\centering
					\caption{$\Psi$:}
					\begin{tabular}{*{9}{c}}
						XQ & 1 & 2 & 3 & 4 & 5 & 6 & 7 & 8 \\
						\hline
						a & 4 & 5 & 3 & 5 & 7 & 1 & 5 & 3 \\
						b & 2 & 1 & 5 & 8 & 2 & 3 & 3 & 5 \\
						c & & & & & & & & \\
					\end{tabular} \\
				\end{table}

				\begin{table}[h]
					\centering
					\caption{$\Phi$:}
					\begin{tabular}{*{9}{c}}
						XQ & 1 & 2 & 3 & 4 & 5 & 6 & 7 & 8 \\
						\hline
						a & 1 & 0 & 0 & 1 & 1 & 1 & 1 & 0 \\
						b & 0 & 1 & 1 & 0 & 0 & 0 & 0 & 1 \\
						c & 1 & 0 & 0 & 0 & 1 & 0 & 0 & 0 \\
					\end{tabular} \\
				\end{table}

				\begin{enumerate}
					\item $P_1: {1,5}, {2,3,8}, {4,6,7}$
					\item $P_2: \overbrace{\{1,5\}}^{q^{\sim}_1}, \overbrace{\{2,3,8\}}^{q^{\sim}_2}, \overbrace{\{2\}}^{q^{\sim}_3}, \overbrace{\{3,8\}}^{q^{\sim}_4}$
					\item $P_3=P_2$
				\end{enumerate}

				\begin{table}[h]
					\centering
					\begin{tabular}{*{5}{c}}
						& $\tilde q_1$ & $\tilde q_2$ & $\tilde q_3$ & $\tilde q_4$ \\
						\hline
						a & $\tilde q_4$ & $\tilde q_1$ & $\tilde q_3$ & $\tilde q_4$ \\
						b & $\tilde q_2$ & $\tilde q_1$ & $\tilde q_1$ & $\tilde q_3$ \\
						c & $\tilde q_1$ & $\tilde q_4$ & $\tilde q_4$ & $\tilde q_4$ \\
					\end{tabular}
				\end{table}

				\begin{table}[h]
					\centering
					\begin{tabular}{*{5}{c}}
						& $\tilde q_1$ & $\tilde q_2$ & $\tilde q_3$ & $\tilde q_4$ \\
						\hline
						a & 1 & 0 & 0 & 1 \\
						b & 0 & 1 & 1 & 0 \\
						c & 1 & 0 & 0 & 0 \\
					\end{tabular}
				\end{table}

		\subsection{Минимизация состояний частичных автоматов}
			\paragraph{}
				В частичных автоматах не все входные последовательности являются допустимыми.
				\begin{equation}
					\Psi: X*xQ \to Q, \\
					\Phi: X*xQ \to Y, \\
					\alpha \in X*, \alpha=\beta x. \\
				\end{equation}
				(``!'' - значение определено)
				$!\Psi(\alpha,q)=\Psi(\beta,x,q) \Leftrightarrow !\Psi(\beta,q) or !\Psi(x,\Psi(\beta,q)) \\
				!\Phi(\alpha,q)=!\Phi(\beta,x,q,) \Leftrightarrow \Phi(\beta,q) or !\Phi(x,\Psi(\beta,q))$

				$ !f(a) $ - значение функции на наборе $a$ определено
				$q \in Q, \phi_q, X^{*} \rightarrow Q$
				Рассмотрим 2 частичные функции $f,g$ и будем говорить, что функция $f$ продолжает функцию $g$, если выполняется условие:
				\begin{enumerate}
					\item Если $ !g(a) \rightarrow !f(a) $
					\item $ g(a) = f(a) $
				\end{enumerate}
				Рассмотрим некоторое $ \phi_q X^{*} \rightarrow Q $

				Покрытие $S$ - подмножество состояний $Q$ такие, что $\cup S_i = Q$
				Подмножество покрытия называют блоками 
				Покрытие называется сохраняемым, если в для каждого блока $\in S$ выполняются условия:
				\begin{enumerate}
					\item $ \forall s_i^1, s_i^2 !\psi(x,s_i^1) \forall x, \\
					 !\psi(x,s_i^1) \psi(x,s_i^1),\psi(x,s_i^2) \subseteq Q $
				\end{enumerate}
				Сохряняемые покрытия называются правильными, если $!\phi(x,s_i^1), !\phi(x,s_i^2) \Rightarrow \phi(x,s_i^1)=\phi(x,s_i^1)$.
				Теорема: Для автомата $ A \rightarrow A^\prime_m \exists $ сохраняемо правильное покрытие минимальной мощности.

				Состояния $q_i,q_j$ будем называть совместимыми, если $!\phi(\alpha,q_i),!\phi(\alpha,q_j) \Rightarrow \phi(\alpha,q_i)=\phi(\alpha,q_j)$
				Подмножество состояний из $q$ будем называть совместимым подмножеством, если $\forall$ пара его состояний совместима
				Подмножество состояний из $q$ будем называть несовместимым подмножеством, если $\forall$ пара его состояний несовместима

				Для пары состояний построим цепь $q_i,q_j$:
				$(q_i,q_j) \in C_{q_i,q_j},$
				Состоянием $q_i,q_j$ назовем 1-совместимыми, если $\forall x !\phi(x,q_i), !\phi(x,q_j) \Rightarrow !\phi(x,q_i)=!\phi(x,q_j)$
				Цепь - совокупность состяний из множества $Q$
				Теорема: $q_i,q_j$ совместимы, если в построенной цепи $C_{q_i,q_j}$ все состояний 1-совместимы

\end{document}