\documentclass[a5paper,10pt]{article}

\usepackage{cmap}
\usepackage[a5paper,margin=1.5cm,noheadfoot]{geometry}
\usepackage[T2A]{fontenc}
\usepackage[utf8]{inputenc}
\usepackage[english,russian]{babel}
\usepackage{blindtext}
\usepackage{mathtools}
\usepackage{framed}
\usepackage{enumitem}
\usepackage{amsthm}
\usepackage{amssymb}
\usepackage{listings}

\author{Евгений Гужихин}
\title{Алгоритмы и методы программирования, 2 курс, 1 семестр ФПМК, 2 контрольная}
\date{\today}

\begin{document}

	\maketitle

	\tableofcontents{}
	\newpage

	\section{Сортировка Хоара (Quicksort)}
		\subsection{Описание}
			Общая идея алгоритма:
			\begin{enumerate}
				\item Выбрать из массива элемент, называемый опорным. Это может быть любой из элементов массива.
				\item Сравнить все остальные элементы с опорным и переставить их в массиве так, чтобы разбить массив на два непрерывных отрезка: "меньшие опорного", "больше".
				\item Для отрезков "меньших" и "больших" значений выполнить рекурсивно ту же последовательность операций, если длина отрезка больше единицы.
			\end{enumerate}
		
		\subsection{Алгоритм}
			\begin{lstlisting}[frame=single]
algorithm quicksort(A, lo, hi) is
  if lo < hi then
	p := partition(A, lo, hi)
	quicksort(A, lo, p – 1)
	quicksort(A, p, hi)

algorithm partition(A, lo, hi) is
  pivot := get_pivot(A, lo, hi)
  i := lo
  j := hi    
  while i < j do
	while A[i] < pivot do
	  i := i + 1 
	while A[j] > pivot do
	  j := j - 1 
	if i <= j then
	  swap A[i] with A[j]
  return i + 1
			\end{lstlisting}
			Предполагается, что массив A передаётся по ссылке, то есть сортировка происходит "на том же месте", а неописанная функция $get\_pivot$ возвращает значение опорного элемента.

		\subsection{Выбор опорного элемента}
			\begin{enumerate}
				\item $ pivot = (max-min)/2 $, если извесны $ min $ и $ max $;
				\item $ pivot = a[(int)(N/2)] $ - выбор среднего элемента;
				\item $ pivot = a[(int)(rand()*N)] $ - выбор случайного элемента.
			\end{enumerate}

		\subsection{Реализация на C}
			\begin{lstlisting}[frame=single,language=C]
template < class T >
void quickSort(T* a, long N) {
  long i = 0, j = N-1;
  T temp, p;

  p = a[ N>>1 ]; // pivot

  // partition
  do {
	while ( a[i] < p ) i++;
	while ( a[j] > p ) j--;

	if (i <= j) {
	  temp = a[i]; a[i] = a[j]; a[j] = temp;
	  i++; j--;
	}
  } while ( i<=j );

  // recursion
  if ( j > 0 ) quickSort(a, j);
  if ( N > i ) quickSort(a+i, N-i);
}
			\end{lstlisting}

		\subsection{Оценка}
			\begin{enumerate}
				\item \textbf{Лучший случай:} $ O(n \log n) $
				\item \textbf{Худший случай:} $ O(n^2) $
			\end{enumerate}

	\section{Поиск медианы}
		\subsection{Описание}
		\subsection{Алгоритм}
		\subsection{Реализация}
		\subsection{Оценка}

	\section{Поразрядная сортировка (Radixsort)}
		\subsection{Описание}
			Рекурсивная сортировка, числа сортируются по разрядам. Cначала сортируются младшие разряды, затем старшие.

		\subsection{Реализация на С}
			\begin{lstlisting}[frame=single,language=C]
template < class T >
void radixSort(T* a, int left, int right, int radix) {
  if (left >= right || radix < 0) return;
  int mask = 1 << radix, i = left, j = right;

  // partition
  while (i <= j) {
	while (i <= j && !(a[i] & mask)) ++i;
	while (i <= j && (a[j] & mask)) --j;
	if (i < j)
	  swap(i++, j--);
  };

  // recursion
  if (left < j) RadixSort(a, left, j, radix-1);
  if (i < right) RadixSort(a, i, right, radix-1);
}
			\end{lstlisting}

		\subsection{Оценка}
			\begin{enumerate}
				\item \textbf{Лучший случай:} $ O(n \log n) $
				\item \textbf{Худший случай:} $ O(n^2) $
			\end{enumerate}
			

\end{document}