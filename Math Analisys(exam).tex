\documentclass{article}

\usepackage{cmap}
\usepackage[a5paper,margin=1.5cm,noheadfoot]{geometry}
\usepackage[T2A]{fontenc}
\usepackage[utf8]{inputenc}
\usepackage[english,russian]{babel}
\usepackage{blindtext}
\usepackage{mathtools}

\begin{document}

	\section{Математический анализ(3 семестр, $\S$1.5 Вычисление определенного интеграла, часть II Замена переменной, Теорема 2)}

	\paragraph{\underline{Теорема:}}
		Пусть:
		\begin{enumerate}
			\item $f(x)$ интегрируема на $[a,b]$;
			\item $\phi(t)$ определена и монотонна на $[\alpha,\beta]$, $\phi(\alpha)=a$, $\phi(\beta)=b$;
			\item $\exists$ непрерывная $\phi^\prime(t)$ на $[\alpha,\beta]$.
		\end{enumerate}
		Тогда $\int\limits^b_a f(x)dx = \int\limits^\alpha_\beta f(\phi(t))\phi^\prime(t)dt$.
	\paragraph{\underline{Доказательство:}}
		\begin{enumerate}

			\item Разобьем $[a,b]$ на части $\alpha<t_1<...<t_{n-1}<\beta$. Введем $\Delta t_i=t_{i+1}-t_i$, $\lambda=\max|\Delta t_i|$, произвольное $\tau_i \in [t_i,t_{i+1}]$.

			\item Составим интегральную сумму для $\int\limits^\alpha_\beta f(\phi(t))\phi^\prime(t)dt: \\
			\sigma=\sum\limits^{n-1}_{i=0}f(\phi(\tau_i))\phi^\prime(\tau_i)\Delta t_i$. Точки $x_i=\phi(t_i), i=\overline{0,n-1}$ разобьют $[a,b]$ на части.

			\item При $\lambda \rightarrow 0$ (все $\Delta t_i \rightarrow 0$), $\phi\uparrow: \Delta x_i=\phi(t_{i+1})-\phi(t_i)=\omega$ - колебание $\phi(t)$ на $[t_i,t_{i+1}]$, т.к. по условию $\phi(t)$ непрерывно на $[\alpha,\beta] \Rightarrow $ (по Т. Кантора) $\phi(t)$ равномерно непрерывна на $[\alpha,\beta] \Rightarrow$ все $\omega_i \rightarrow_{\lambda \rightarrow 0} 0$, $i=\overline{0,n-1}$, т.е. $\lambda \rightarrow 0 \Rightarrow \max\limits_i \Delta x_i \rightarrow 0$, причем по ф-ле Лагранжа $\Delta x_i=\phi(t_{i+1})-\phi(t_i)=\phi\prime(\xi)\Delta t_i$.

			\item Тогда при $\xi_i = \phi(\tau_i)$ интегральная сумма $\sigma\prime$ для $\int\limits^b_af(x)dx: \\
			\sigma\prime=\sum\limits^{n-1}_{i=0}f(\phi(\tau_i))\phi\prime(\hat\tau)\Delta t_i = \sum\limits^{n-1}_{i=0}f(\xi_i)\Delta x_i$.

			\item Покажем, что $\lim\limits_{\lambda \rightarrow 0}\sigma=\lim\limits_{\lambda \rightarrow 0}\sigma\prime$, т.е. $\forall \epsilon>0 \exists\delta>0 \forall\lambda<\delta |\sigma-\sigma\prime|<\epsilon L(\beta-\alpha), |f(x)|\leq L$ на $[a,b]$.

			\item Так как $\phi\prime(t)$ равномерно непрерывна (по Т. Кантора) на $[\alpha,\beta] \Rightarrow \left|\phi\prime(\hat\tau_i)-\phi\prime(\tau_i)\right|\leq\omega_i<\epsilon, \forall i = \overline{0,n-1},$ а $\left|f(\phi(\tau_i))\right| \leq L \Rightarrow \\
			\left|\sigma\prime-\sigma\right| = \left|\sum\limits^{n-1}_{i=0} f(\phi(\tau_i))\left|(\phi\prime(\hat\tau)-\phi(\tau_i))\right|\Delta t_i\right| \leq \sum\limits^{n-1}_{i=0}\left|f(\phi(\tau_i))\right|\left|\phi\prime(\hat\tau_i)-\phi\prime(\tau_i)\right|\Delta t_i \leq \epsilon L(\beta-\alpha) \Rightarrow \lim\limits_{\lambda \rightarrow 0}(\sigma\prime-\sigma) = 0$. \fbox{Ч.Т.Д}

		\end{enumerate}

\end{document}