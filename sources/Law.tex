\documentclass[a5paper,10pt]{article}

\usepackage{cmap}
\usepackage[a5paper,margin=1.5cm,noheadfoot]{geometry}
\usepackage[T2A]{fontenc}
\usepackage[utf8]{inputenc}
\usepackage[english,russian]{babel}
\usepackage{blindtext}
\usepackage{mathtools}
\usepackage{framed}
\usepackage{enumitem}
\usepackage{amsthm}
\usepackage{amssymb}

\author{Евгений Гужихин}
\title{Право(конспект), 2 курс, 2 семестр ФПМК}
\date{\today}

\begin{document}

	\maketitle
	\tableofcontents{}
	\newpage

	\section{Понятия и сущность государственного права}
		\subsection{Понятие государства и его признаки}
			\textbf{Теории происхождения государства:}
			\begin{enumerate}
				\item \underline{Божественная теория:} появление государства связано с волей божьей;
				
				\item \underline{Материалистическая теория:} рост производствеенных сил, развитие орудий и средств производства, что повлекло появление имущественных излишек, а так же расслоение общества;
				
				\item \underline{Психологическая теория:} причина появления государства не в экономике, а в особенности человеческой психики, в умении человека приспосабливаться к действительности, тяга к объединению;
				
				\item \underline{Теория насилия:} ключевая роль государства - захват одних народов другими. Для удержания власти захватчик создает государство и право;
				
				\item \underline{Патриархальная теория:} государство возникает в результате разрастания семьи в патриархальную общину. Рода и премена образуют государство;
				
				\item \underline{Органическая теория:} государство - проявление сил природы. Оно появляется вместе с людьми. Отдельные части человеческого организма соответствуют отдельным частям государства;
				
				\item \underline{Теория общественного договора:} народ - носитель ествественных и неотъемленных прав, которые принадлежат им от рождения. Государство образуется в тот момент, когда население пошло на добровольное ограничение своих прав для упорядочивания всех отношений к общей пользе.
			\end{enumerate}

			\textbf{Государство} - форма политической организации общества, осуществляющаяся на определенной территории в отношении проживающего на ней населения, характеризующаяся наличием государственного аппарата, который реализует функции управления на основе закрепленных законов, норм права в целях обеспечения частных и публичных интересов и наделенная возможностью применения государственного принуждения.

			\textbf{Признаки государства:}
			\begin{enumerate}
				\item \underline{Территория} - ограниченное пространство, в пределах которого распространяется суверенитет государства, и где органы государственной власти осуществляют свои полномочия;
				
				\item \underline{Население} - человеческое сообщество, проживающее на территории государства;
				
				\item \underline{Публичная власть} - способность, возможность и право определять общественное поведение и деятельность населения, проживающего в этом государстве;
				
				\item \underline{Налоги} - общеобязательные и безвзмездные платежи, взыскиваемые в заранее установленных размерах и в определенные сроки, необходимые для содержания органов управления и поддержания жизнедеятельности государства;
				
				\item \underline{Наличие права} - система общеобязательных, законодательно закрепленных правил поведения, которые издаются в форме законов и иных актов;
				
				\item \underline{Наличие армии} - вооруженные силы, выполняющие защиту государства от внешних угроз;
				
				\item \underline{Государственный язык};
				
				\item \underline{Гражданство};
				
				\item \underline{Государственная символика} - герб, гимн, флаг;
				
				\item \underline{Единая денежная система};
				
				\item \underline{Междунароное признание}.
			\end{enumerate}

	\section{Форма государства}
		\subsection{Форма правления}
			\textbf{Форма правления} - организация высших органов государства, порядок образования и взаимоотношения, степень участия граждан в их формировании.

			По формам правления государства подразделяются на: \textbf{монархии} и \textbf{республики}.

			В \textbf{монархиях} полномочия главы государства - монарха передаются по наследству и осоществляются им единолично и пожизненно.

			В \textbf{неограниченных монархиях} единственным носителем суверенитета государства является монарх.

			В \textbf{ограниченных монархиях} наряду с монархом, носителем суверенности государства являются другие высшие государственные органы, ограничивающие власть главы государства.

			\textbf{Республика} - форма правления, при которой главой государства является избранный на определенный срок, единоличный или коллективный орган.

			Республики бывают: \textbf{парламентские} и \textbf{президентские}.

			В \textbf{парламентской республике} президенту принадлежит, в основном, символические функции. Исполнительная власть принадлежит правительству во главе с премьер-министром. Правительство формируется на основе парламентского большинства.

			В \textbf{президентской республике} президент является главой государства и правительства.

		\subsection{Формы государственного устройства}
			\textbf{Форма государственного устройства} - внутренняя структура государства, способ его политического и территориального деления, обуславливающий определенные взаимоотношения органов всего государства с органами его составных частей.

			\textbf{Формы государственного устройства:}
			\begin{enumerate}
				\item \underline{Унитарное государство} - простое цельное, единое государство, части которого являются административно-территориальными единицами и не обладают признаками государственного суверенитета. В этом государстве существует единая система высших органов и единая система законодательства.
				
				\item \underline{Федеративное государство} - сложное союзное государство, части которого являются отдельными государствами или государственными образованиями, обладающие в той или иной мере государственным суверенитетом, признаками государственности. В нем, наряду с высшими органами федерации и федеральным законодательством, существуют высшие органы и законодательство частей, членов федерации.
				
				\item \underline{Конфедерация} - союз государств, образуемый для определенных целей. Объединение двух или более государств, при котором союзные органы лишь координируют деятельность государств и только по тем вопросам, для разрешения которых они объединились. Отсутствие общего центрального государственного аппарата. Каждое государство сохраняет за собой право вступать в дипломатические, торговые и военные соглашения с иностранными государствами.
			\end{enumerate}

		\subsection{Государственно-правовой режим}
			\textbf{Государственно-правовой режим} - совокупность средств и способов осуществления политической власти, выражающих ее содержание и характер.
			\begin{framed}
				Например: деспотия, аристократия, фашизм, демократия.
			\end{framed}

			\begin{enumerate}			
				\item \textbf{Демократический государственно-правовой режим} - государственная власть, основанная на принципе подчинения меньшинства большинству, где решения принимаются большинством с одновременным уважением прав меньшинства.

				\item \textbf{Авторитарный государственно-правовой режим} - осуществление государственной власти в противоречии с демократическими нормами и институтами, реализация правителем власти по своему произволу, без учета мнения большинства.
			\end{enumerate}

	\section{Функции государства}
		\subsection{Внутренние функции}
			\textbf{Внутренние функции} - основные направления деятельности государства по управлению внутренней жизнью страны.

			\begin{enumerate}
				\item \textbf{Экономическая функция} - выражается в выработке и координации государством стратегий и тактики развития страны в наиболее оптимальном режиме.
				\begin{framed}
					Экономическая деятельность проявляется в установлении налогов, высоты кредитов, инвестиций, определении льгот, строитеельстве инфраструктуры.
				\end{framed}

				\item \textbf{Политическая функция} - обусловлена необходимостью осуществления государством гармонизации интересов различных социальных групп.

				\item \textbf{Социальная функция} - ее выполнение позволяет обеспечить нормальные условия жизни для всех членов общества вне зависимости от их участия в производстве материальных благ, а так же от возраста, пола, здоровья и т.д. Установление минимального уровня оплаты труда, пенсий, стипендий, пособий и т.д. Социальная политика проводится за счет налогов.

				\item \textbf{Правоохранительная функция} - деятельность государства по обеспечению точного и полного выполнения положений законодательства всеми гражданами, органихациями, государственными органами.

				\item \textbf{Экологическая функция} - регулирует деятельность людей и организаций в области использования окружающей среды.

				\item \textbf{Функции обеспечения национальной безопасности}, \textbf{строительства дорог}, \textbf{транспортные}, \textbf{информационные}...
			\end{enumerate}

		\subsection{Внешние функции}
			\textbf{Внешние функции} - основные напрвления деятельности государства на международной арене.
			\begin{enumerate}
				\item \textbf{Функция обороны} - регулирование территориальных притязаний и разрешение конфликтов между государствами.

				\item \textbf{Дипломатическая функция} - способствует поддержанию приемлимых отношений со всеми странами.

				\item \textbf{Внешнеполитическая функция} - заключается в политическом сотрудничестве государств с тем, чтобы исключить глобальные вооруженные конфликты.

				\item \textbf{Внешнеэкономическая функция} - связана с развитием взаимовыгодного сотрудничества государств, которое проявляется в международном разделении труда, специализации и кооперировании производства, обмена новейшими технологиями, развитии кредитно-финансовых связей.

				\item \textbf{Культурное сотрудничество} - заключается на основе двусторонних и многосторонних договоров между государствами, неправительственными организациями.

				\item \textbf{Сотрудничество государств в решении глобальных проблем современности} - рациональное использование природных ресурсов, охрана окружающей среды, сохранение энергии, поддержание мира и т.д.
			\end{enumerate}

	\section{Понятия и сущность права}
		\textbf{Право} - системная совокупность исходящих от государства норм, имеющих общественное значение.
		\subsection{Признаки права:}
			\begin{enumerate}
				\item Все нормы исходят от государства и имеют законодательное закрепление;

				\item Право имеет одинаковое значение для всех субъектов, либо для отдельной группы субъектов;

				\item Государство регулирует только общественные отношения;

				\item Праву присуще определенные формы закрепления: либо закон, либо подзаконный акт.
			\end{enumerate}

		\subsection{Функции права}
			\begin{enumerate}
				\item \textbf{Регулятивная:} все нормы права имеют цель урегулировании общественных отношений;

				\item \textbf{Охранительная:} охраняет общественный порядок;

				\item \textbf{Воспитательная:} воспитывает благополучных граждан путем устрашения или воспитания.
			\end{enumerate}

		\subsection{Отрасли права}
			Каждая норма права имеет составные части:
			\begin{enumerate}
				\item \underline{гипотеза} - часть нормы, указывающая при каких обстоятельствах она действует ("если ...");

				\item \underline{диспозиция} - правило поведения, которое исходит из объекта;

				\item \underline{санкция} - ответственность за нарушение диспозиции ("иначче ...").
			\end{enumerate}

		\subsection{Виды норм права}
			Все нормы права делятся:
			\begin{itemize}[itemsep=0pt]
				\item \underline{по отрослям права}
					\begin{itemize}[itemsep=0pt]
						\item административные;
						\item гражданские;
						\item трудовые;
						\item уголовные;
						\item семейные;
						\item и т.д.
					\end{itemize}

				\item \underline{по функциям}
					\begin{itemize}[itemsep=0pt]
						\item регулятивные;
						\item охранительные.
					\end{itemize}
				
				\item \underline{по характеру правил} поведения, которые содержатся в нормах
					\begin{itemize}[itemsep=0pt]
						\item обязывающие нормы;
						\item запрещающие нормы;
						\item управомачивающие нормы.
					\end{itemize}

				\item \underline{по кругу лиц}, на которые разпространяется норма права
					\begin{itemize}[itemsep=0pt]
						\item общие нормы;
						\item специальные нормы.
					\end{itemize}
				
				\item \underline{по юридической силе}
					\begin{itemize}[itemsep=0pt]
						\item законы;
						\item подзаконные нормативные акты.
					\end{itemize}
				
				\item \underline{по действию в пространстве}
					\begin{itemize}[itemsep=0pt]
						\item на всей территории РФ;
						\item на отдельных частях РФ.
					\end{itemize}
			\end{itemize}

			Основной элемент права - \textbf{отрасль права} - совокупность норм, регулирующих однородные общественные отношения.
	
	\section{Основы конституционного строя}
		\textbf{Конституция} - основной закон государства, обладающий высшей силой, закрепляющий и регулирующий основные моменты по защите прав и свобод личности, а также функции государственной власти.

		\begin{framed}
			Конституция РФ принята \underline{12.12.1993 г.} всенародным голосованием.
		\end{framed}

		Состоит из \underline{преамбулы} и \underline{9 глав}:
		\begin{itemize}[itemsep=0pt]
			\item[\underline{1 глава:}] "Основы конституционного строя";
			\item[\underline{2 глава:}] "Права и свободы человека и гражданина";
			\item[\underline{3 глава:}] "Федеративное устройство";
			\item[\underline{4 глава:}] "Президент РФ";
			\item[\underline{5 глава:}] "Федеральное собрание";
			\item[\underline{6 глава:}] "Правительство";
			\item[\underline{7 глава:}] "Судебная власть";
			\item[\underline{8 глава:}] "Местное самоуправление";
			\item[\underline{9 глава:}] "Конституционные поправки и пересмотр конституции".
		\end{itemize}

		\begin{framed}
			Уровни власти:
			\begin{itemize}[itemsep=0pt]
				\item государственная власть;
				\item федеральная власть;
				\item органы власти субъектов;
				\item органы местного самоуправления.
			\end{itemize}
		\end{framed}

		\textbf{Основы конституционного строя} - правопорядок, при котором соблюдается демократическая конституция.

		В каждой конституции декларируется защита прав и свобод.

		\underline{Россия, как:}
		\begin{enumerate}
			\item \textbf{демократическое государство}
				\begin{itemize}[itemsep=0pt]
					\item единственным источником власти является народ;
					\item власть осуществляется в соответствии с волей большинства при соблюдении и охране прав меньшинства;
					\item власть регулируется посредством демократических процедур.
				\end{itemize}

			\item \textbf{правовое государство}
				\begin{itemize}[itemsep=0pt]
					\item признает права и свободы как высшую ценность;
					\item закрепляется верховенство закона надо любыми общественными событиями;
					\item устанавливает принципы осуществления государственной власти.
				\end{itemize}

			\item \textbf{республиканская форма правления государства}
				\begin{itemize}[itemsep=0pt]
					\item власть осуществляют только выборные органы;
					\item переход к монархической форме правления возможен только путем принятия поправок в конституцию, при этом посягательство на республиканскую форму правления является антиконституционным и преследуется по закону;
					\item монархическая форма правления запрещена во всех субъектах РФ.
				\end{itemize}

			\item \textbf{социальное государство} (с 1991, в РФ)
				\begin{itemize}[itemsep=0pt]
					\item государство равных возможностей для человека и гражданина, позволяющее любому развиваться, учиться, выражать свободу вероисповедания и т.д.
				\end{itemize}

			\item \textbf{светское государство}
		\end{enumerate}

		\subsection{Федеративное устройство}
			\begin{itemize}[itemsep=0pt]
				\item \underline{государственная целостность};
				\item \underline{единство системы} госуарственной власти;
				\item \underline{равноправие} и \underline{самоопределение} народов РФ.
			\end{itemize}

		\subsection{Субъекты РФ}
			\underline{84 субъекта:} 22 республики, 9 краев, 45 областей, 3 города федерального значения(Москва, Санкт-Петербург, Севастополь), 1 автономная область, 4 автономных округа.

			Состав РФ закреплен в конституции (статья 65), в которой перечислены все субъекты РФ.
			(15 марта 2014г. - присоединение р. Крым и г. Севастополя)

		\subsection{Система органов государственной власти}
			\underline{Уровни системы:}
			\begin{itemize}[itemsep=0pt]
				\item органы государственной власти РФ;
				\item органы государственной власти субъектов.
			\end{itemize}

			Государственную власть на федеральном уровне представляют:
			\begin{itemize}[itemsep=0pt]
				\item президент РФ
				\item федеральное собрание
					\begin{itemize}
						\item совет федерации
						\item государственная дума
					\end{itemize}
				\item правительство РФ
				\item федеральные судьи
			\end{itemize}

		\subsection{Правовой статус президента РФ}
			Президент не включается ни в одну из ветвей власти. Согласно конституции, президент является главой государства, обладает неприкосновенностью, имеет штандарт, знак президента и специально изданный официальный текст конституции.

			Президенту принадлежат обширные полномочия почти во всех сферах деятельности:
			\begin{enumerate}
				\item \underline{полномочия} по формированию о обеспечению нормального функционирования федеральных органов государственной власти (назначает председателя правительства РФ, федеральных министров и снимает их с должностей. Назначает выборы в государственную думу. Предоставляет совету федерации кандидатуры судей и генерального прокурора)

				\item \underline{полномочия в сфере деятельности} (имеет право издавать указы нормативного характера, подписывает и обнародует федеральные законы, имеет право законадательной инициативы)

				\item \underline{полномочия} в области внешней политики, обороны и национальной безопасности (осуществляет руководство внешней политикой, ведет переговоры и подписывает международные договоры, является верховным главнокомандующим ВС РФ, утверждает военную доктрину, концепцию национальной безопасности, объявляет призыв граждан на военную службу, вводит военное или чрезвычайное положение на всей территории РФ или в отдельных субъектах)

				\item \underline{полномочия в сфере конкретизации статуса личности} (решает вопросы гражданства, предоставляет политическое убежище, награждает государственными наградами, присваивает почетные звания, высшие воински и высшие специальные звания, осуществляет помилования)
			\end{enumerate}

			Президенту помогает сформированная им \underline{администрация президента}.
			
			\underline{Государственный совет}, возглавляемый президентом, создан для нормального функционирования государственной власти.

			\underline{Совет безопасности}, возглавляемый президентом, создан для осуществления согласования государственной политики в области безопасности страны.

			В федеральных округах действуют полномочные представители при президенте.

			\underline{Президент} избирается гражданами РФ на основе всеобщего, равного и прямого избирательного права при тайном голосовании. Участие гражданина в выборах является добровольным.

			\underline{Главой государства может быть избран} гражданин не моложе 35 лет, постоянно проживающий на территории РФ не менее 10 лет.

			\underline{Не имеет права быть избранным в президенты} гражданин РФ, имеющий гражданство другой страны или имеющий вид на жительство, лицо осужденное за тяжкие преступления или преступления экстремистской направленности, если на момента выбора не погашена судимость.

			Президент вступает в должность по истечении 6 лет со дня вступлении в должность предыдущего президента, а при досрочных выборах на 30-ый день со дня публикации результатов.

			При вступлении в должность он приносит присягу.

			Согласно конституции РФ, \underline{полномочия} президента \underline{прерываются} по следующим основаниям:
			\begin{enumerate}
				\item истечение срока пребывания в должности;
				\item в случае отставки (добровольного сложения полномочий в силу политических, личных, либо иных обстоятельст);
				\item в случае стойкой неспособности по состоянию здоровья осуществлять принадлежащие ему полномочия;
				\item в случае отрешения от должности на основании обвинения в государственной измене или совершения иного тяжкого преступления.
			\end{enumerate}

			Исполняющий обязанности президента - председатель правительства РФ.

		\subsection{Законодательные органы государственной власти}
			\begin{enumerate}
				\item \textbf{Совет федерации} по 2 представителя от каждого субъекта
				
				\item \textbf{Государственная дума} (450 депутатов) - нижняя палата федерального собрания, избираемая сроком на 5 лет \par
					(Члены совета федерации и депутаты государственной думы заседают раздельно. Один и тот же человек не может быть одновременно депутатом и членом совета федерации) \par
					\textbf{Депутатом} государственной думы может быть избран гражданин РФ, достигший 21 года и имеющий право участвовать в выборах.
					Согласно конституции, депутаты работают на постоянной основе и не могут занимать должности на государственной службе, заниматься другой оплачиваемой деятельностью, кроме преподавательской, научной и творческой. \par
					\underline{Полномочия:}
					\begin{itemize}
						\item принятие к рассмотрению и принятие государственных законов;
						\item принятие государственного бюджета;
						\item назначение председателя счетной палаты;
						\item объявление амнистии.
					\end{itemize}
					Собирается на первое собрание на 30-ый день после избрания. Заседание открывает старейший по возрасту. Избирается председатель и заместитель.

				\item \textbf{Федеральное собрание} - постоянно действующий орган государственной власти, избираемый сроком на 5 лет \par
					\underline{Его полномочия:}
					\begin{itemize}
						\item утверждает указы президента о вводе военного положения и чрезвычайного положения
						\item решает вопрос о возможном использовании ВС РФ за пределами территории РФ
						\item назначет выборы президента
						\item назначает и освобождает от должности генерального прокурора
						\item и т.д.
					\end{itemize}
			\end{enumerate}

		\subsection{Органы исполнительной власти}
			\begin{itemize}[itemsep=0pt]
				\item \underline{Правительство РФ}
				\item Федеральное министерство
				\item Государственные комитеты
				\item Федеральные службы
				\item Федеральные надзоры
				\item Департаменты
				\item Главные управления
				\item Агенства
			\end{itemize}

			\textbf{Правительство} - возглавляет исполнительную власть, является коллегиальным органом.
			
			\underline{Состав:}
			\begin{itemize}[itemsep=0pt]
				\item Председатель
				\item Заместители
				\item Федеральные министры
			\end{itemize}

			\underline{Функции:}
			\begin{itemize}[itemsep=0pt]
				\item распределение деятельности в сфере образования, науки, культуры
				\item обеспечивает законные права граждан
				\item реализация функций во внутренней и внешней политики
				\item формирует и исполняет бюджет
				\item и т.д.
			\end{itemize}

			У каждого субъекта РФ есть свои органы, как законодательные, так и исполнительные.

	\section{Понятие, задачи и принчипы уголовного права}
		\textbf{Метод} - совокупность приемов и способов регулирования, составляющего предмет уголовного права.

		\textbf{Либеративный метод} - носит гос-властный характер и выражается применением к виновным.

		Такие отношения, как установление факта преступления и назначение наказания, осуществляется в порядке регламента.

		Решение о применении к лицу ответственности решается только судьей.

		\subsection{Задачи уголовного законодательства}
			\begin{itemize}
				\item Охрана прав и свобод, собственности, общественного порядка, окружающей среды, конституционного строя от преступных посягательств
				\item Обеспечение мира и безопасности человека
				\item Предупреждение преступления
			\end{itemize}

		\subsection{Принципы уголовного права}
			\begin{enumerate}
				\item \textbf{Законность} - преступные деяния, наказуемость и т.д. определяется у.к.
				\item \textbf{Равенство всех перед законом} - все, независимо от пола, расы и т.д. равны перед законом и подлежат равной ответственности
				\item \textbf{Ответственность за вину} - лицо подлежит уголовной ответственности только за те общественно-опасные действия(бездействия) и наступившие общественно-опасные последствия, в отношении которых установлена его вина. Ведение ответственности за невинное приченение вреда недопускается.
				\item \textbf{Справедливость} - наказания, применяемые к лицу, совершившему преступление, должно быть справедливым, т.е. соответствовать характеру и степени общественной опасности преступления, обстоятельством его совершения и личности виновного. Никто не несет ответственности дважды за одно и то же преступление.
				\item \textbf{Гумманизм} - наказание не направленно на приченение физического или морального вреда.
			\end{enumerate}

		\subsection{Источники уголовного законодательства}
			\begin{itemize}[itemsep=0pt]
				\item уголовный кодекс РФ
				\item конституция РФ
				\item международные правовые акты
			\end{itemize}

		\subsection{Уголовный кодекс}
			\textbf{Общая часть:} понятийный аппарат(6 разделов, 17 глав и статьи 1-104)
			\textbf{Особенная часть:} нормы о конкретных преступлениях(19 глав, 6 разделов, статьи 105-361)

			\underline{Общаяя часть:}
			\begin{enumerate}[itemsep=0pt]
				\item Уголовный кодекс (2 главы)
				\item Преступление (6 глав)
				\item Наказание (2 главы)
				\item Освобождение от уголовной ответственности и от наказания (3 главы)
				\item Уголовная ответственность несовершеннолетних (1 глава)
				\item Принудительные меры медицинского характера (1 глава)
			\end{enumerate}

			\underline{Особенная часть:}
			\begin{enumerate}[itemsep=0pt]
				\item Преступления против личности (5 глав)
				\item Преступления в сфере экономики (3 главы)
				\item Преступления против общественной безопасности и общественного порядка (5 глав)
				\item Преступленя против государственной власти (4 главы)
				\item Преступления против воинской службы (1 глава)
				\item Преступления против мира и безопасности человека (1 глава)
			\end{enumerate}

		\subsection{Виды преступлений}
			\begin{enumerate}[itemsep=0pt]
				\item небольшой тяжести (не более 2 лет)
				\item средней тяжести (не более 3 лет)
				\item тяжкие (не более 10 лет)
				\item особо тяжкие (от 10 лет)
			\end{enumerate}

		\subsection{Обстоятельства, исключающие преступные деяния}
			\begin{itemize}[itemsep=0pt]
				\item необходимая оборона
				\item приченение вреда при задержании преступника
				\item крайняя необхожимость
				\item физическое/писхическое принуждение
				\item обоснованный риск
				\item исполнение приказа/распоряжения
			\end{itemize}

		\subsection{Состав преступления}
			Для привлечения лица, совершившего преступление, необходимо установить, что есть состав преступления(совокупность, установленных законодательством, объективных и субъективных признаков характеризующих общественно-опасное деяние как преступление)

			\textbf{Объект преступления} - охраняемые законом, личностные, общественные и государственные интересы, которым может быть причинен вред в результате преступного посягательства.

			\textbf{Объективная сторона} - выражается в общественно-опасном деянии, создающем приченение вреда объекту, охраняемому уголовным законом.

			\underline{Признаки:}
			\begin{enumerate}[itemsep=0pt]
				\item причинно-следственная связь между деянием и преступлением
				\item средства(орудия), место, время преступления
			\end{enumerate}

			\textbf{Средства} - физическое лицо, совершившее преступление с определенного возраста.

			\textbf{Признаки:} вменяемость, способность отдавать отчет и руководить своими действиями.

	\section{Основы трудового права}
		\textbf{Трудовое право} - отрасль права, регулирующая общественные отношения, складывающиеся в сфере применения профессиональных знаний, способностей, умений и навыков физических лиц в процессе функционированя хозяйственных субъектов.

		\textbf{Принципы трудового права} - руководящие идеи, которые выражают сущность, основные свойства трудового право.

		Все \underline{принципы} закреплены в \underline{трудовом кодексе}:
		\begin{itemize}
			\item \textbf{свобода труда} - любой гражданин имеет право самостоятельно распоряжаться своими навыками, выбирать род занятий;
			\item \textbf{запрет принудительного труда} и дискриминации в сфере труда;
			\item \textbf{защита от безработицы} и содействие в трудоустройстве;
			\item \textbf{равенство прав} и возможностей работников;
			\item \textbf{право на} своевременную и в полном размере выплату \underline{заработной платы};
			\item обеспечение права любому на \textbf{защиту} его \underline{трудовых прав};
			\item \textbf{обязательность возмещения вреда}, причиненного работнику в связи с выполнением его трудовых обязанностей;
			\item и т.д.
		\end{itemize}

		\textbf{Трудовой договор} - соглашение между работником и работодателем, в соответствии с которым работодатель обязуется:
		\begin{itemize}
			\item предоставить работнику работу по обусловленной трудовой функции;
			\item обеспечить условия труда, предусмотренные трудовым кодексом и иными нормативными актами;
			\item своевременно и в полном размере выплачивать заработную плату.
		\end{itemize}
		Работник обязуется выполнять определенную этим соглашением трудовую функцию, соблюдать действующие внутри организации правила внутреннего трудового распорядка.

		\textbf{Содержание трудового договора} - условия, установленные работодателем, условия, установленные трудовым кодексом.

		\textbf{Необходимые условия трудового договора} - персонификация: в шапке договора ФИО работника, наименование организации, дата заполнения, место работы, структурное подразделение, трудовая функция, время начала работы, оплата труда.

		\textbf{Срок} трудового договора не ограничен, но есть и срочные трудовые договора на срок менее 5 лет.

		\textbf{Испытательный срок} обычно составляет 3 месяца, но может быть и другим.

		Трудовой договор устанавливает перечень документов для заключения:
		\begin{itemize}[itemsep=0pt]
			\item паспорт(либо иной документ, удостоверяющий личность)
			\item трудовая книжка(исключения: первый рабочий опыт, работа по совместительству)
			\item страховое свидетельство государственного пенсионного страхования
			\item документы воинского учета
			\item документы об образовании, квалификации и наличии специальных знаний
			\item иные документы(мед. книжка, ...)
		\end{itemize}

		После подписания трудового договора работодатель обязан в трехжневный срок издать приказ о приеме на работы, в котором содержатся условия трудового договора:
		\begin{itemize}[itemsep=0pt]
			\item место работы;
			\item трудовая функция;
			\item дата начала работы;
			\item заработная плата.
		\end{itemize}

		Трудовой договор вступает в силу со дня подписания, если не указан другой срок. Если работник по истечению 10 дней не вступил в обязанности без уважительной причины, договор аннулируется. После издания приказа кадровая служба вносит запись в трудовую книжку.

		\textbf{Трудовая книжка} - основной документа о трудовой деятельности и стаже. В нее вносится записи о работе, переводах, повышениях, награждениях(Сведения о взысканиях, кроме увольнения, не ведется)

		\subsection{Основания прекращения трудового договора}
			\begin{itemize}[itemsep=0pt]
				\item по соглашении сторон;
				\item истечении срока трудового договора;
				\item по инициативе работника;
				\item по инициативе работодателя;
				\item перевод работника по его просьбе(согласию) к другому работодателю;
				\item отказ работника от продолжении работы в связи со сменой собственника имущества организации, либо ее реорганизации;
				\item отказ работника от продолжении работы в связи с изменением существенных условий трудового договора;
				\item отказ работника от перевода на другую работу в следствии состояния здоровья в соответствии с медицинским заключением
				\item отказ работника от перевода, в связи с перемещением работодателя в другую местность;
				\item независящие от сторон;
				\item нарушения, установленные, в соответствии с трудовым кодексом или иными законами.
			\end{itemize}

		\subsection{Трудовые правоотношения}
			\textbf{Трудовые правоотношения} - добровольная юридическая связь работника с работодателем по поводу применения знаний, умений, способностей и навыков работника в процессе труда.

			\underline{Стороны трудового правоотношения}
			\begin{itemize}[itemsep=0pt]
				\item[-] работник
				\item[-] работодатель
			\end{itemize}

		\subsection{Права работника}
			\begin{itemize}[itemsep=0pt]
				\item на заключение, изменение, расторжение трудового договора;
				\item на предоставление ему работы, обусловленной трудовым договором;
				\item право на рабочее место, соответствующее всем стандартам и нормам безопасности труда;
				\item право на своевременную и в полном объеме выплату заработной платы;
				\item право на отдых;
				\item право на полную и достоверную информацию об условиях труда и требованиях охраны труда на рабочем месте;
				\item право на проф. подготовку, переподготовку, повышение квалификации;
				\item на ведение коллективных переговоров, создание профсоюзов.
			\end{itemize}

		\subsection{Обязанности работника}
			\begin{itemize}[itemsep=0pt]
				\item добросовесное исполнение трудовых обязанностей, возложенных на нег трудовым договором;
				\item соблюдение правил внутреннего трудового распорядка;
				\item соблюдение трудовой дисциплины;
				\item свыполнение установленной нормы труда;
				\item соблюдение требований по охране труда и обеспечение безопасности труда;
				\item бережно относиться к имуществу работодателя и других работников;
				\item незамедлительно сообщать работодателю о возникновении ситуации, представляющей угрозу жизни и здоровью людей, сохранности имущества работодателя.
			\end{itemize}

		\subsection{Права работодателя}
			\begin{itemize}[itemsep=0pt]
				\item заключать, изменять, расторгать трудовой договор;
				\item вести коллективные переговоры, заключать коллективные договора;
				\item поощрять работника за добросовесный и эффективный труд;
				\item требовать от работников исполнения их трудовых обязанностей и бережного отношения к имуществу работодателя;
				\item привлекать работников к дисциплинарной и материальной ответственности;
				\item принимать локальные нормативные акты;
				\item создавать обхединения работников.
			\end{itemize}

		\subsection{Обязанности работодателя}
			\begin{itemize}[itemsep=0pt]
				\item соблюдение закона и других нормативно-правовых актов;
				\item предоставлять работу, обусловленную трудовым договором;
				\item обеспечивать безопасность труда и условия, отвечающие такой безопасности;
				\item обеспечивать оборудованием, инструментами и иными средствами;
				\item обеспечивать работникам равную оплату за равную работу;
				\item выплачивать в полном размере заработную плату;
				\item вести коллективные переговоры;
				\item совевременно выполнять предписания государственных надзорных и контрольных органов, уплачивать штрафы.
			\end{itemize}

		\subsection{Признаки трудового правоотношения}
			\begin{itemize}[itemsep=0pt]
				\item волевое;
				\item длящееся;
				\item двухстороннее;
				\item индивидуальное;
				\item возмездное.
			\end{itemize}

\end{document}