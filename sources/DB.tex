\documentclass[a5paper,10pt]{article}

\usepackage{cmap}
\usepackage[a5paper,margin=1.5cm,noheadfoot]{geometry}
\usepackage[T2A]{fontenc}
\usepackage[utf8]{inputenc}
\usepackage[english,russian]{babel}
\usepackage{blindtext}
\usepackage{mathtools}

\author{Евгений Гужихин}
\title{Базы данных(конспект), 3 курс, 1 семестр ФПМК}
\date{\today}

\begin{document}

	\maketitle
	\tableofcontents{}
	\newpage

	\section{1 семинар}
		\subsection{Реляционная алгебра}
			\paragraph{Определение отношения}	
				\textbf{Отношение} - подмножество декатового произведения $ D_1 * D_2 * ... * D_n $
				Множества $ D_i $ - \textbf{домены}
				Элементы декартова произведения $ d_1 * d_2 * ... * d_n $ - \textbf{кортежи}
				Число $ n $ - \textbf{степень отношения}
				Количество кортежей $ (|R|) $ - \textbf{мощность кортежа}

			\paragraph{Основные компоненты реляционного отношения}

			\paragraph{Свойства отношений}
				\begin{itemize}
					\item Отсутствие кортежей-дубликатов
					\item Отсутствие упорядоченности кортежей
					\item Отсутствие упорядоченности атрибутов
					\item Атомарность значений атрибутов
				\end{itemize}

		\subsection{}
			\paragraph{\textbf{S}tructured \textbf{Q}uery \textbf{L}anguage}
				\begin{itemize}
					\item \textbf{DDL} - Data Defenition Language --- язык определения данных
					\item \textbf{DML} - Data Manupulation Language --- язык манипулирования данными
				\end{itemize}

			\paragraph{Select}
				\textbf{DISTINCT} - зло на проде(долгое)
				\textbf{SELECT [ALL | DISTINCT]} <имя\_столбца|перечень\_столбцов>
				\textbf{FROM} <имя\_таблицы>, ...
					\textbf{JOIN} <имя\_таблицы> \textbf{ON} <условие>
				\[ \textbf{WHERE} <условие> \]
				\[ \textbf{GROUP BY} <имя\_столбца> \]
				\[ \textbf{HAVING} <условие> \]
				\[ \textbf{ORDER BY} <имя\_столбца>, ... \[\] \]

			\paragraph{Логические операции и операции сравнения}
				\textbf{LIKE} - Поиск по заданному шаблону
				\textbf{BETWEEN} - Принадлежность диапазону
				\textbf{IN} - Принадлежность списку
			
			\paragraph{Поиск по шаблону (\textbf{LIKE})}
				\% - любое количество символов (в том числе и 0) подряд
				\_ - любой одиночный символ
				\textbf{ESCAPE} - поиск специальных символов (\% и \_).
				ESCAPE определяет символ после которого любой следующий за ним символ интерпретируется как обычный.
				\underline{Пример:}  

			\paragraph{NULL}
				\begin{itemize}
					\item Сравнение: \begin{lstlistening} IS \[NOT\] NULL \end{lstlistening}
					\item \textbf{NULL} и пустая строка эквивалентны
				\end{itemize}

		\subsection{Связи}
			\paragraph{Отношения}
				\begin{itemize}
					\item Один к одному (1:1) - "хорошие отношения"
					\item Один ко многим (1:N) - "плохие отношения"
					\item Многие ко многим (N:M) - "свободные отношения"
				\end{itemize}
			
			\paragraph{JOIN}
				\textbf{CROSS JOIN}	- внешнее соединение (декартово произведение элементов нескольких таблиц)
				\textbf{FULL JOIN} - полное соединение
				\textbf{INNER JOIN} - внутреннее соединение(отстутствующие связи выкидываются), "пересечение множеств"
				\textbf{LEFT JOIN} - расширение левой таблицы(левая таблица - главная, из нее данные не выкидываются)
				\textbf{RIGHT JOIN} - расширение правой таблицы

			\begin{lstlistening}
				DROP TABLE <table_name>;
				DELETE FROM ;
			\end{lstlistening}

\end{document}
