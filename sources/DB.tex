\documentclass[a5paper,10pt]{article}

\usepackage{cmap}
\usepackage[a5paper,margin=1.5cm,noheadfoot]{geometry}
\usepackage[T2A]{fontenc}
\usepackage[utf8]{inputenc}
\usepackage[english,russian]{babel}
\usepackage{blindtext}
\usepackage{mathtools}

\author{Евгений Гужихин}
\title{Базы данных(конспект), 3 курс, 1 семестр ФПМК}
\date{\today}

\begin{document}

	\maketitle
	\tableofcontents{}
	\newpage

	\section{1 семинар}
		\subsection{Реляционная алгебра}
			\paragraph{Определение отношения}	
				\textbf{Отношение} - подмножество декатового произведения $ D_1 * D_2 * ... * D_n $
				Множества $ D_i $ - \textbf{домены}
				Элементы декартова произведения $ d_1 * d_2 * ... * d_n $ - \textbf{кортежи}
				Число $ n $ - \textbf{степень отношения}
				Количество кортежей $ (|R|) $ - \textbf{мощность кортежа}

			\paragraph{Основные компоненты реляционного отношения}

			\paragraph{Свойства отношений}
				\begin{itemize}
					\item Отсутствие кортежей-дубликатов
					\item Отсутствие упорядоченности кортежей
					\item Отсутствие упорядоченности атрибутов
					\item Атомарность значений атрибутов
				\end{itemize}

		\subsection{}
			\paragraph{\textbf{S}tructured \textbf{Q}uery \textbf{L}anguage}
				\begin{itemize}
					\item \textbf{DDL} - Data Defenition Language --- язык определения данных
					\item \textbf{DML} - Data Manupulation Language --- язык манипулирования данными
				\end{itemize}

			\paragraph{Select}
				\textbf{DISTINCT} - зло на проде(долгое)
				\textbf{SELECT [ALL | DISTINCT]} <имя\_столбца|перечень\_столбцов>
				\textbf{FROM} <имя\_таблицы>, ...
					\textbf{JOIN} <имя\_таблицы> \textbf{ON} <условие>
				\[ \textbf{WHERE} <условие> \]
				\[ \textbf{GROUP BY} <имя\_столбца> \]
				\[ \textbf{HAVING} <условие> \]
				\[ \textbf{ORDER BY} <имя\_столбца>, ... \[\] \]

			\paragraph{Логические операции и операции сравнения}
				\textbf{LIKE} - Поиск по заданному шаблону
				\textbf{BETWEEN} - Принадлежность диапазону
				\textbf{IN} - Принадлежность списку
			
			\paragraph{Поиск по шаблону (\textbf{LIKE})}
				\% - любое количество символов (в том числе и 0) подряд
				\_ - любой одиночный символ
				\textbf{ESCAPE} - поиск специальных символов (\% и \_).
				ESCAPE определяет символ после которого любой следующий за ним символ интерпретируется как обычный.
				\underline{Пример:}  

			\paragraph{NULL}
				\begin{itemize}
					\item Сравнение: \begin{lstlistening} IS \[NOT\] NULL \end{lstlistening}
					\item \textbf{NULL} и пустая строка эквивалентны
				\end{itemize}

		\subsection{Связи}
			\paragraph{Отношения}
				\begin{itemize}
					\item Один к одному (1:1) - "хорошие отношения"
					\item Один ко многим (1:N) - "плохие отношения"
					\item Многие ко многим (N:M) - "свободные отношения"
				\end{itemize}
			
			\paragraph{JOIN}
				\textbf{CROSS JOIN}	- внешнее соединение (декартово произведение элементов нескольких таблиц)
				\textbf{FULL JOIN} - полное соединение
				\textbf{INNER JOIN} - внутреннее соединение(отстутствующие связи выкидываются), "пересечение множеств"
				\textbf{LEFT JOIN} - расширение левой таблицы(левая таблица - главная, из нее данные не выкидываются)
				\textbf{RIGHT JOIN} - расширение правой таблицы

			\begin{lstlistening}
				DROP TABLE <table\_name>;
				DELETE FROM ;
			\end{lstlistening}

			\paragraph{Параметры инициализации}
				\textbf{NLS\_SESSION\_PARAMETERS} - таблица параметров

				\begin{lstlistening}
					CASE [expression]
						WHEN condition\_1 THEN result\_1
						...
						WHEN condition\_n THEN result\_n
						ELSE result\_def
					END [AS] result
				\begin{lstlistening}
				\textbf{AS} - псевдоним

	\section{2 семинар}
		\subsection{Типы данных}
			\paragraph{Символьные типы данных}
				\begin{table}[]
					\centering
					\caption{My caption}
					\label{my-label}
					\begin{tabular}{lcc}
					 & Фиксированная длина & Переменная длина \\
					 Традиционные типы(набор символов базы данных) & CHAR[{N}] & VARCHAR2(N) \\
					 Юникод(Национальный набор символов) & NCHAR[{N}] & NVARCHAR2(N) \\
					\end{tabular}
				\end{table}

				\begin{lstlistening}
					VARCHAR2 (<макс\_длина> [CHAR | BYTE])
					CHAR [(<макс\_длина> [CHAR | BYTE])]
				\begin{lstlistening}

				\textbf{CHAR} - максимальная длина выражается в символах
				\textbf{BYTE} - максимальная длина выражается в байтах

				\underline{Параметры инициализации:}
				\begin{enumerate}
					\item \textbf{NLS\_LENGTH\_SEMANTICS} для измерения максимальной длины
					\item \textbf{MAX\_SQL\_STRING\_SIZE} максимальный размер строк с Oracle
				\end{enumerate}

				CAST('123', VARCHAR2(10)) AS BLA
				CONCAT('abs', 'de') AS A1
				CONCAT(a, b) = a || b
				CONCAT('test', NULL) => 'test'

			\paragraph{Числовые типы данных}
				\textbf{NUMBER [(X,[Y])]} - применяется для хранения целых, чисел с плавующей точкой и т.д.
				X - всего знаков, Y - знаков после запятой

			\paragraph{Дата и время}
				\begin{lstlistening}
					EXTRACT (
						{ YEAR | MONTH | DAY | HOUR | MINUTE | SECOND }
						| { TIMEZONE\_HOUR | TIMEZONE\_MINUTE }
						| { TIMEZONE\_REGION TIMEZONE_ABBR }
					)
					FROM ...
				\end{lstlistening}

			\paragraph{Большие типы данных}
				BFILE
				BLOB
				CLOB
				NCLOB
				XMLType

	\section{3 семинар}	
		\subsection{Подзапросы}
			\begin{itemize}
				\item \textbf{Некоррелированные (незаписимые) запросы} - логически выполняются ровно один раз
				\item \textbf{Коррелированные (связанные) запросы} - отличается от независимых тем, что его значение зависит от переменной, получаемой от внешнего запроса
			\end{itemize}

			\paragraph{WITH}
				\begin{framed}
					Работает иногда быстрее, чем подзапросы.
				\end{framed}
				\begin{lstlistening}
					WITH \[NAME\] as
					\{
						SELECT statement
					\}
					SELECT \[column list\]
					FROM \[NAME\]
				\end{lstlistening}
			
			\paragraph{UNION, INTERSECT, MINUS}
				Работают с таблицами, как с множествами. (Объединение, пересечение, вычитание)
				\begin{framed}
					Должно быть одинаковое количество столбцов и одинаковые типы!
				\end{framed}
				\textbf{UNION ALL} - не удаляет одинаковые записи
				\begin{lstlistening}
					(SELECT ... FROM ... WHERE ...)
					UNION
					(SELECT ... FROM ... WHERE ...)
					ORDER BY raw\_number
					GROUP BY ...
					HAVING ...
				\end{lstlistening}

			\paragraph{Оконные функции}
				\begin{lstlistening}
					SELECT PRODUCT\_ID, PRODUCT\_NAME, PRICE,
						   RANK() OVER(PARTITION BY CATEGORY\_ID ORDER BY PRICE DECS) AS "rank"
					FROM PRODUCTS;
				\end{lstlistening}

		\subsection{DDL. Дополнительные элементы СУБД Oracle.}
			\paragraph{Транзакции}
				\textbf{Транзакция} - последовательность команд SQL, которую Oracle, интерпретирует как единый рабочий блок.
				\underline{Основные концепции:}
				\begin{enumerate}
					\item \textbf{Атомарность:} любая транзакция будет зафиксирована только полностью
					\item \textbf{Согласованность:} любая завершенная транзакция фиксирует только допустимые результаты
					\item \textbf{Изолированность:} каждая транзакция должна быть изолированная от других. (На практике крайне труднодостежимая вещь)
					\item \textbf{Долговечность:} после получения подтверждения о выполнении транзакции, изменения, вызванные этой транзакцией, не должны быть отменены из-за сбоя системы
				\end{enumerate}
				\underline{Команды:}
				\begin{itemize}
					\item \textbf{COMMIT} - сохраняет изменения
					\item \textbf{ROLLBACK} - откатывает изменения
					\item \textbf{SAVEPOINT} <sp\_name> - устанавливает контрольную точку с заданным именем, которая помечает текущую точку в транзакции
					\item \textbf{ROLLBACK TO} <sp\_name> - откатывает изменения на контрольную точку
				\end{itemize}

		\subsection{DDL и DML}
			\begin{framed}
				Любая команда DDL завершает текущую транзакцию
			\end{framed}
			\paragraph{Создание таблиц}
				\begin{lstlistening}
					CREATE TABLE <table\_name>
					( <raw\_name>, <raw\_type>
					\[NOT NULL\]
					\[UNIQUE | PRIMARY KEY\]
					\[REFERENCES <master\_table\_name> \[<raw\_name\]\]
					, ... )
				\end{lstlistening}

			\paragraph{Изменение схемы таблицы}
				\begin{lstlistening}
					ALTER TABLE <table\_name> (ADD ... | DROP ... | MODIFY)
				\end{lstlistening}

			\paragraph{Переименование и удаление таблиц}
				\begin{lstlistening}
					RENAME TABLE <table\_name> ON <new\_table\_name>;
					DROP TABLE <table\_name>;
				\end{lstlistening}
				

\end{document}
