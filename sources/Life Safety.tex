\documentclass[a5paper,10pt]{article}

\usepackage{cmap}
\usepackage[a5paper,margin=1.5cm,noheadfoot]{geometry}
\usepackage[T2A]{fontenc}
\usepackage[utf8]{inputenc}
\usepackage[english,russian]{babel}
\usepackage{blindtext}
\usepackage{mathtools}
\usepackage{framed}
\usepackage{enumitem}
\usepackage{amsthm}
\usepackage{amssymb}
\usepackage{hyperref}

\author{Евгений Гужихин}
\title{Безопасность жизнедеятельности. Глобальные проблемы окружающей среды. Проблемы климата. 2 курс, 2 семестр ФПМК}
\date{\today}

\begin{document}

	\maketitle
	\newpage

	\begin{abstract}
		\item \paragraph{Характеристика темы и исследуемой проблемы}
			В данном реферате рассмотрены глобальные проблемы окружающей среды, а именно проблема климата, его изменение, причины изменения, а так же проблема парниковых газов. Влияние различных факторов на возникновение парниковых газов и последствия для деятельности человека и окружающей среды. Проблема окружающей среды является одной из актуальнейших проблем человечества по сей день, как для хозяйственной деятельности человечества на данный момент, так и для выживаемости человечества в будущем.
	\end{abstract}

	\newpage
	\tableofcontents{}
	\newpage

	\section{Изменение климата}
		\textbf{Изменение климата} - колебания климата Земли в целом или отдельных её регионов с течением времени, выражающиеся в статистически достоверных отклонениях параметров погоды от многолетних значений за период времени от десятилетий до миллионов лет. Учитываются изменения как средних значений погодных параметров, так и изменения частоты экстремальных погодных явлений. Изучением изменений климата занимается наука палеоклиматология. Причиной изменения климата являются динамические процессы на Земле, внешние воздействия, такие как колебания интенсивности солнечного излучения, и, с недавних пор, деятельность человека. Изменения в современном климате (в сторону потепления) называют глобальным потеплением.\cite{wiki}
 		
 		Cовременная проблема в метеорологии - глобальные изменения климата, возможность его прогнозирования на большие сроки. То, что за последние 150 лет происходит изменение термического режима атмосферы, не вызывает никакого сомнения. Происходит глобальное потепление атмосферы - примерно на 1-1,5 градуса. Особенно интенсивно в последние 20-25 лет. Но оно имеет свои региональные и временные масштабы. Наиболее заметно на территории России потеплел климат в умеренных широтах Европейской России и в Западной Сибири зимой. зимы в последнее время стали "сиротскими". летом температурный режим практически не изменился. А в южных районах, в частности, на Украине, даже несколько похолодало. То же в северных регионах - Архангельской области, Республике Коми - климат совсем не потеплел. Есть периоды времени, когда это потепление наиболее заметно себя проявляет. Потеплел климат Аляски, а вот климат благодатной Калифорнии несколько похолодал. Дать однозначное этому объяснение довольно сложно, хотя эта проблема сейчас активно изучается во многих странах, поскольку дальнейшее потепление на земном шаре может привести к весьма негативным последствиям. Уменьшится количество ледников в северных морях (например, в Гренландии), что приведет к подъему уровня Мирового Океана, тогда окажутся под водой прибрежные территории, уровень которых ниже уровня моря. Это, например, Нидерланды, которые под натиском моря только с помощью дамб сохраняют свою территорию; Япония, у которой в таких районах находятся многие производственные мощности; могут быть залиты океаном многие острова в тропиках. Но произойдет ли это - вопрос весьма дискуссионный. атмосфера может потеплеть еще на 1 градус через ближайшие 100 лет, но утверждать это мы не можем в настоящее время.\cite{meteonovosti}

		\subsection{Естественные и техногенные причины изменения климата}

		\newpage

	\section{Парниковые газы}
		%\cite[p.~215]{lamport94}
		\subsection{Влияние промышленности на этот процесс}


		\subsection{Последствия для окружающей среды и человека}
			Изменения климата становятся очевидными и принимают всё более угрожающий характер. Временные масштабы климатических изменений на планете, безусловно, превышают среднестатистическую продолжительность «политической жизни» тех лиц, которые принимают решения, касающиеся безопасности и судеб целых народов. Ныне мировая политика потребительского общества всё более утрачивает маску человеческого лица, обнажая свою подлинную суть. Достаточно рассмотреть вопрос, какие на сегодняшний день принимаются меры по обеспечению безопасности народов тех или иных странах и кто в реальности обеспечивает себе безопасность, прикрываясь «заботой о народе».

			Политика определённых международных организаций и развитых стран, некоторые учёные, спонсируемые ими, поддерживают теорию, что одной из основных причин изменения глобального климата на Земле является антропогенное воздействие на природу, связанное с выбросом парниковых газов в атмосферу. На этом основании разработаны различные международные документы, такие, например, как Киотский протокол (дополнение к Рамочной конвенции ООН об изменении климата). Однако практика показала неэффективность подобных документов…

			По иронии судьбы такая причина как «антропогенное влияние», несомненно, имеет место быть, но носит исключительно политический и коммерческий характер. Вместо ожидаемого массами исполнения правителями оглашённых намерений по улучшению климатической ситуации на планете, на практике выполнение этих обязательств было превращено в коммерческий проект, торговлю квотами, и привело лишь к обогащению отдельных заинтересованных лиц. К сожалению, эти международные документы стали лишь аргументом в торговых войнах и фактором оказания давления на экономическую политику той или иной страны. Они преследовали в большей степени бизнес-интересы определённых лиц, чем реальную попытку что-то улучшить на планете. Увы, снова сработал чисто человеческий фактор, доминирование решений, продиктованных не самыми лучшими человеческими побуждениями отдельных лиц.

			К сожалению, в нашем мире потребительского общества оглашаются и массово популяризируются такие климатические гипотезы, которые на самом деле выгодны лишь тем странам, которые их инициируют, поддерживают и пропагандируют. У одних стран - это политический интерес, у других - экономический. А в целом - утопический взгляд на кардинальное решение вопросов, связанных с глобальным изменением климата. Но зато вполне реальное осуществление стратегического соперничества, скрытой борьбы за власть и мировое влияние, что увеличивает риск конфронтации между мировыми державами. Как известно из теории систем, любая идея, приносящая высокие доходы, используется во всё более сложных условиях до тех пор, пока не станет причиной большой катастрофы...

			Без сомнения человеческая деятельность в масштабах планеты негативно влияет на окружающую среду. Но это влияние минимально по сравнению с тем, что происходит на планете в результате влияния комплекса природных факторов, которое в ближайшем будущем будет только нарастатьи о чём не перестают вещать добропорядочные учёные мира. На сегодняшний день антропогенное воздействие не является причиной массовых планетарных катаклизмов по указанным выше причинам. Глобальные климатические изменения на Земле происходят по независимым от человечества обстоятельствам и требуют реальной консолидации усилий всех людей на планете для выживания цивилизации в ближайшем будущем, и об этом стоит задуматься каждому жителю нашей планеты.

			Масштабные природные катаклизмы, циклично происходящие на планете, уже неоднократно случались в истории Земли и человеческой цивилизации. Но какие уроки преподносят эти научные знания, свидетельствующие о былых всеобщих планетарных трагедиях? Природные катаклизмы не имеют «государственных границ», этих искусственно созданных условностей, которые были придуманы правителями для разделения и власти над людьми. Последствия и беды, что приносят общепланетарные катаклизмы, выходят далеко за «очаговое» отдельно взятое государство и, так или иначе, касаются всех жителей Земли. Резкое повышение сейсмической и вулканической активности приводит к мгновенным катастрофическим последствиям в тех или иных регионах. Исчезают с лица Земли целые государства, гибнут люди, многие остаются без крова и средств к существованию, начинаются голод и широкомасштабные эпидемии.\cite{allatra}

		\newpage

	\section{Выводы}

		\newpage

	\section{Источники}
		\begin{thebibliography}{9}
			\bibitem{wiki} Wikipedia, \url{https://ru.wikipedia.org/wiki/Изменение_климата}, 06.06.2017
			\bibitem{meteonovosti} Meteonovosti, \url{http://www.meteonovosti.ru/index.php?index=14&value=4}, 06.06.2017
			\bibitem{allatra} «О проблемах и последствиях глобального изменения климата на Земле. Эффективные пути решения данных проблем», Кристина Ковалевская, Координационный центр Международного общественного движения «АЛЛАТРА»
		\end{thebibliography}

\end{document}